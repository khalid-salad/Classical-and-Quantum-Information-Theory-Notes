\section{Entropy}

\begin{definition}[Entropy]
    \label{def:entropy}
    Let $\Pr$ be a probability distribution on a discrete space $\Omega$. The
    Shannon Entropy (hereby simply Entropy) of $\Pr$ is defined
    \[\entropy{\Pr} = \sum_{\omega \in \Omega}\prob{\omega}\log\frac{1}{\prob{\omega}}\]
    If $X$ is a discrete random variable, we define
    \begin{align*}\entropy{X}
         & = \entropy{\Pr_X}                                    \\
         & = \sum_{x \in X}\prob[X]{x}\log\frac{1}{\prob[X]{x}} \\
         & = \expectation{\log\frac{1}{\prob[X]{X}}}
    \end{align*}
    noting that $\log\frac{1}{\prob[X]{X}}$ is a real random variable.
\end{definition}

We can think of $\log\frac{1}{\prob[X]{x}}$ as the level of ``surprise'' that
$X = x$ occurs and $\entropy{X}$ as the uncertainty or randomness of $\Pr_X$.

Note that, in \cref{def:entropy}, $\log$ refers to $\log_2$, and $\log_2(X)$ is
the number of bits of $X$. Additionally, since a byte is 8 bits, $\log_{256}(X)$
is the number of bytes of $X$. Additionally, we define $0\log\frac{1}{0} = 0$,
which can be motivated by the fact that

\[\lim_{x\to0^+}x\log\frac{1}{x} = 0\]

\begin{example}[Bernoulli Distribution]
    The Bernoulli Distribution is the discrete random variable
    \begin{align*}
        \prob{X = 1} & = p     \\
        \prob{X = 0} & = 1 - p
    \end{align*}
    and has entropy
    \[\entropy{X} = p\log\frac{1}{p} + (1 - p)\log\frac{1}{1 - p}\]
\end{example}

\begin{definition}[Binary Entropy]
    The binary entropy of $p$, $\binentropy{p}$, is the entropy of the Bernoulli Distribution with
    parameter $p$, i.e.,
    \[\binentropy{p} = p\log\frac{1}{p} + (1 - p)\log\frac{1}{1 - p}\]
\end{definition}

Notice that $\binentropy{0} = \binentropy{1} = 0$ and $\binentropy{\frac{1}{2}} = 1$.
More generally, the graph of $\binentropy{p}$ is given in \cref{fig:binentryopy}.

\begin{figure}
    \centering
    \tikzsetnextfilename{binentryopy}
    \begin{tikzpicture}[scale=1]
        \begin{axis}[
                axis lines = left,
                x label style={at={(axis description cs:0.5,-0.1)}},
                xlabel = $p$,
                ylabel = $\binentropy{p}$,
                xmin = 0,
                ymin = 0,
                xmax = 1,
                ymax = 1,
                xtick = {0.5},
                xticklabels = {$\sfrac{1}{2}$},
                ytick = {1},
                % xmajorticks=false,
                % ymajorticks=false,
            ]
            \addplot [
                domain=0:1,
                samples=10000,
                color=red,
                thick,
            ]
            {- x * ln(x) / ln(2) + -(1 - x) * ln(1 - x) / ln(2)}
            node [pos=0.5, above] (m) {most random};
        \end{axis}
    \end{tikzpicture}
    \todo[inline]{add some flourishes}
    \caption{Binary Entropy as a function of $p$. Notice that the entropy is
        maximized when $p = \sfrac{1}{2}$ and 0 when $p = 0$ or $p = 1$. When $p = 0$
        or $p = 1$, the Bernoulli Distribution is non-random, and thus there is no
        uncertainty.}
    \label{fig:binentryopy}
\end{figure}

\begin{example}[Geometric Distribution]
    The Geometric Distribution is the positive, integer-valued random variable
    that describes the number of Bernoulli trials performed until a success. That
    is,
    \[\prob{X = k} = p(1 - p)^{k - 1}\]
    is the probability that it will require $k$ trials until a success.

    The entropy of the Geometric Distribution is given by
    \begin{align*}\entropy{X}
         & = \sum_{k = 1}^{\infty} p(1 - p)^k\log\frac{1}{p(1 - p)^k}                                              \\
         & = \sum_{k = 1}^{\infty} p(1 - p)^k\left(\log\frac{1}{p} + k\log\frac{1}{1 - p}\right)                   \\
         & = p\log\frac{1}{p}\sum_{k = 1}^{\infty} (1 - p)^k + p\log\frac{1}{1 - p}\sum_{k = 1}^{\infty}k(1 - p)^k \\
         & = p\frac{1}{p}\log\frac{1}{p} + p\log\frac{1}{1 - p}\frac{1 - p}{p^2}                                   \\
         & = \frac{1}{p}\left(p\log\frac{1}{p} + (1 - p)\log\frac{1}{1 - p}\right)                                 \\
         & = \frac{\binentropy{p}}{p} \to 0 \hbox{ as $p \to 0^+$}
    \end{align*}
\end{example}

\begin{example}[Distribution with \infty~Entropy]
    TBD
\end{example}

An empirical justification for the use of $\log_2$.

\todo[inline]{Finish figure, caption, and description.}
\begin{figure}
    \caption{Drawing of ant nest used to empirically verify \dots}
\end{figure}

\begin{definition}[Convexity]
    Let $V \isomorphic \reals^n$ be a vector space. A subset $S \subseteq V$ is
    convex if, for any pair $\vec{x}, \vec{y} \in S$
    \[\lambda\vec{x} + (1 - \lambda)\vec{y} \in S\hbox{ for all $\lambda \in [0, 1]$}\]
\end{definition}

\todo[inline]{figure demonstrating convexity}

\begin{example}
    The following are convex
    \begin{enumerate}[label=(\arabic*)]
        \item $\reals^n$
        \item
        \item
    \end{enumerate}
\end{example}

\begin{definition}[Convex Function]
    A function $f: S \to \reals$ is
    \begin{enumerate}[label=(\roman*)]
        \item convex if $f\left(\lambda\vec{x} + (1 - \lambda)\vec{y}\right) \leq \lambda f(\vec{x}) + (1 - \lambda)f(\vec{y})$ for all $\vec{x}, \vec{y} \in S$ and $\lambda \in [0, 1]$
        \item \emph{strictly} convex if $f\left(\lambda\vec{x} + (1 - \lambda)\vec{y}\right) < \lambda f(\vec{x}) + (1 - \lambda)f(\vec{y})$ for all $\vec{x}, \vec{y} \in S$ and $\lambda \in [0, 1]$
    \end{enumerate}
\end{definition}

\begin{definition}[Concave Function]
    A function $f: S \to \reals$ is (strictly) concave if $-f$ is (strictly) convex.
\end{definition}

\begin{example}
    Notice
    \begin{enumerate}[label=(\arabic*)]
        \item The function $x \to x\log{x}$ is strictly convex
        \item The function $x \to \log{x}$ is strictly concave
        \item The function $X \to \expectation{X}$ is convex (but not strictly)
    \end{enumerate}
\end{example}

\begin{theorem}[Jensen's Inequality]
    Let $X$ be a real vector valued random variable. Then, if $f$ is any convex
    function,
    \[f\left(\expectation{X}\right) \leq \expectation{f(X)}\]
    If $f$ is strictly convex, then $f\left(\expectation{X}\right) = \expectation{f(X)}$
    if and only if $X = \expectation{X}$, i.e., $X$ is a constant random variable.
\end{theorem}

\begin{proof}
    Since $f$ is convex,
    \begin{align*}f\left(\expectation{X}\right)
         & = f\left(\sum_{x \in X}x\prob{X = x}\right)                                                  \\
         & \leq \sum_{x \in X}f(x)\prob{X = x}\hbox{ since $f$ is convex and $\prob{X = x} \in [0, 1]$} \\
         & = \expectation{f(X)}\qedhere
    \end{align*}
\end{proof}

\begin{theorem}[Properties of Entropy]
    The Entropy function satisfies
    \begin{enumerate}[label=(\arabic*)]
        \item $\entropy{X} \geq 0$ with equality if and only if $X$ is constant
        \item if $X$ is finite, then $\entropy{X} \leq \log\card{\mathcal{X}}$ with
              equality if and only if $\Pr_X$ is uniform on $\mathcal{X}$
        \item For any injective $f$, $\entropy{X} = \entropy{f(X)}$
        \item $\Pr \to \entropy{\Pr}$ is strictly concave
    \end{enumerate}
\end{theorem}

\begin{proof}\mbox{}

\end{proof}