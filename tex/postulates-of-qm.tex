\section{Postulates of Quantum Mechanics}

Direct's Bra-cket Notation.

Given a vector $u = \begin{pmatrix}u_1\\u_2\\\vdots\\u_n\end{pmatrix}\in\complexnums^n$,
we write $\ket{u}$, prounounced ``ket'' $u$.

For $v \in (\complexnums)^* = \linop{\complexnums^n}{\complexnums}$, We denote by $\bra{v}$, proununced ``bra'' $v$,
\[v = \begin{pmatrix}\conj{v_1}&\conj{v_2}&\dots&\conj{v_n}\end{pmatrix} = \begin{pmatrix}v_1\\v_2\\\vdots\\v_n\end{pmatrix}^*\]

Notice that
\begin{align*}\bra{u}\ket{v}
     & = \begin{pmatrix}\conj{u_1}&\conj{u_2}&\dots&\conj{u_n}\end{pmatrix}\begin{pmatrix}v_1\\v_2\\\vdots\\v_n\end{pmatrix} \\
     & = \sum_{i=1}^n\conj{u_i}v_i                          \\
     & = \innerproduct{u}{v}
\end{align*}
hence we denote it $\bracket{u}{v}$.

Similarly,

\[\underset{\linop{\complexnums}{\complexnums^n}}{\ket{u}^*} = \underset{\linop{\complexnums^n}{\complexnums}}{\bra{u}}\]

Finally, write
\[\ket{1} = \begin{pmatrix}1\\0\\\vdots\\0\end{pmatrix}\quad\ket{2} = \begin{pmatrix}1\\2\\\vdots\\0\end{pmatrix} \dots\]

How it works:

\begin{enumerate}[label=\arabic*)]
    \item Inner Product: $\bracket{u}{v}$
    \item Matrix Action: $\bra{u}A\ket{v} = \innerproduct{u}{Av}$
    \item Rank one operator: $E_{u,v} = \underset{\hbox{outer product}}{\ket{u}\bra{v}}$ with
          \begin{align*}E_{u, v}\left(\ket{w}\right)
               & = \left(\ket{u}\bra{v}\right)\ket{w} \\
               & = \ket{u}\bracket{v}{w}              \\
               & = \bracket{v}{w}\ket{v}
          \end{align*}
    \item General Operator: $A = \sum a_i\ket{u_i}\bra{v_i}$

          e.g. \begin{align*}A
               & = \begin{bmatrix}a_{ij}\end{bmatrix}     \\
               & = \sum_{ij}a_{ij}\ket{i}\bra{j}
          \end{align*}
          Notice that $\set{\ket{i}}$ are the elements of the standard basis.
    \item Matrix multiplication:
          \begin{align*}
              A  & = \sum a_i\ket{u_i}\bra{v_i}                             \\
              B  & = \sum b_{j}\ket{w_j}\bra{x_j}                           \\
              AB & = \sum_{i,j} a_ib_j \ket{u_i}\bracket{v_i}{w_j}\bra{x_j} \\
                 & = \sum_{i,j}a_ib_j \bracket{v_i}{w_j}\ket{u_i}\bra{x_j}
          \end{align*}
          Notice that $\bracket{v_i}{w_j} \in \complexnums$.
    \item Diagonal operator: $\underset{\hbox{orthogonal decomposition}}{A = \sum\lambda_i\ket{v_i}\bra{v_i}}$
          with eigenvalues $\lambda_i$ and eigenvectors $\ket{v_i}$ of $A$.
\end{enumerate}

Recall the Spectral Theorem
\begin{enumerate}
    \item $A$ positive $\implies$ $\lambda_i \geq 0$
    \item $A$ Hermitian $\implies$ $\lambda_i \in \reals$
    \item $A$ projection $\implies$ $\lambda_i \in \set{0, 1}$
    \item $A$ unitary $\implies$ $\abs{\lambda_i} = 1$ for all $i$
\end{enumerate}

Postulates are working assumptions under a framework or model (see Feynmann's
vide on difference between mathematics and physics \url{https://www.youtube.com/watch?v=obCjODeoLVw}).

\begin{tikzpicture}
    \node (rw) {Real World};
    \node[right=4cm of rw] (qm) {Quantum Mechanics using Mathematics};
    \draw[->] (rw) to node[pos=0.5,above](p) {Postulates} (qm);
    \node[below=0.5cm of p] (exp) {Experiments/Theory};
    \draw[->] (exp.north) -- (p.south);
\end{tikzpicture}

\begin{enumerate}[label=Postulate \arabic*.]
    \item (State Space) Any isolated (quantum) system is associated a $\complexnums$
          Hilbert Space as its \emph{state space}. The state of a system is given by a unit vector
          $\ket{\phi}$ with $\norm{\ket{\phi}} = 1$, called the \emph{state vector}.

          For example, $\complexnums^2 = \set{a\ket{0} + b\ket{1}}$.

          The basis $\ket{0}, \ket{1}$ is called the \emph{logic} basis. Another basis
          is $\ket{+} = \frac{\ket{0} + \ket{1}}{\sqrt{2}}$, $\ket{-} = \frac{\ket{0} - \ket{1}}{\sqrt{2}}$.

          Superposition: $\ket{\phi} = a\ket{0} + b\ket{1}$, unit/state $\iff \abs{a}^2 + \abs{b}^2 = 1$.

          $\ket{\phi} = e^{i\delta}\cos\frac{\theta}{2}\ket{0} + e^{i(\delta+\psi)}\sin\frac{\theta}{2}\ket{1}$ describes
          the Bloch Sphere, with $0 \leq \delta \leq 2\pi$, $0 \leq \theta \leq \pi$, and
          $0 \leq \psi \leq 2\pi$. $e^{\i\delta}$ is the global phase, not physically relevant.
          $\psi$ is the relative phase, physical.

          %TODO draw bloch sphere

          Example: 1-dim wave function %TODO finish 1-dim wave function, double slit, etc.

    \item (Measurement) Quantum Information Form --- every quantum measurement is given
          by a collection $\set{E_m}$ of measurement operators satisfying the completeness
          equation $\sum E_m = I$ with $E_m \geq 0$. The set $\set{E_m}$ is called a
          Positive Operator-Valued Measurement (POVM).

          The probability of seeing an outcome $m$ is given by \[\bra{\phi}E_m\ket{\phi}\]
          with
          \[\sum_{m}\bra{\phi}E_m\ket{\phi} = 1\]
          for any $\phi$. Notice this implies $\sum E_m = I$.

          Pose measurement state: \[\frac{E_m\ket{\phi}}{\norm{E_m\ket{\phi}}} = frac{E_m\ket{\phi}}{\sqrt{\bra{\phi}E_m\ket{\phi}}}\]

          Example: Shr\:{o}dinger's Cat %TODO add figure

          Set $\ket{\phi} = \frac{1}{\sqrt{2}}\ket{0} + \frac{1}{\sqrt{2}}\ket{1}$,
          $E_0 = \ket{0}\bra{0}$, $E_1 = \ket{1}\bra{1}$. After measurement, $\frac{1}{2}$
          prob, $E_0\ket{\phi} = \frac{1}{\sqrt{2}}\ket{0}$, $E_1\ket{\phi} = \frac{1}{\sqrt{2}}\ket{1}$.

          A \emph{projection value measurement} (PVM) is a set of mutually orthogonal projections
          $\set{P_m}$ such that $\sum P_m = 1$. For example, taking $\set{\ket{v_i}}$ to be some
          orthonormal basis, the set $\set{E_i = \ket{v_i}\bra{v_i}}$ is a PVM.

          Quantum Mechanics version: every observable corresponds to a Hermitian operator ($\inf$-dim).
          The only values that will be observed are eigenvalues.

          \begin{example}[Position Operator]
              The position operator is defined
              \[\phi(x) \to x\phi(x)\]
              For example,
              $e^{-\frac{x^2}{2}} \to xe^{-\frac{x^2}{2}}$
              has expected position $\bra{\phi}X\ket{phi}=\int_{-\infty}^{\infty}X\abs{\phi(x)^2}dx$
              \begin{tikzpicture}[scale=1]
                  \begin{axis}[
                          axis lines = left,
                          x label style={at={(axis description cs:0.5,-0.1)}},
                          xlabel = $p$,
                          ylabel = $\binentropy{p}$,
                          xmin = -10,
                          ymin = 0,
                          xmax = 10,
                          ymax = 1,
                          % xmajorticks=false,
                          % ymajorticks=false,
                      ]
                      \addplot [
                          domain=-10:10,
                          samples=10000,
                          color=red,
                          thick,
                      ]
                      {exp(-x^2 / 2)};
                  \end{axis}
              \end{tikzpicture}
          \end{example}
          We can similarly define the momentum operator $P = -i\planck\frac{\partial}{\partial x}$, where
          \planck is the Planck Constant, and the kinetic energy operator
          \[H = -\frac{\planck^2}{2m}\frac{\partial}{\partial x^2}\] where $m$ is the
          mass.

          In finite dimensional space, $A = A^*$ (Hermitian) implies $A = \sum \lambda_i E_I$
          with $\set{E_i}$ a PVM. If $\ket{\phi_i}$ is an eigenvector of $\lambda_i$, then
          $\bra{\phi_i}A\ket{\phi_i} = \lambda_i$, hence all $\lambda_i$ are real.
    \item (Evolution) The evolution of a closed quantum system is described by a unitary, $\ket{\phi} \to U\ket{\phi}$.

          For example,

          \begin{align*}
              X & = \begin{bmatrix}0 & 1\\1&0\end{bmatrix}\quad X\ket{0} = \ket{1} \quad X\ket{1} = \ket{1} \hbox{ bit flip}                        \\
              Z & = \begin{bmatrix}1 & 0\\0&-1\end{bmatrix}\quad Z\ket{0} = \ket{0} \quad Z\ket{1} = -\ket{1} \hbox{ phase flip}                     \\
              H & = \frac{1}{\sqrt{2}}\begin{bmatrix}1 & 1\\1&-1\end{bmatrix}\quad H\ket{0} = \ket{+} \quad H\ket{1} = \ket{-} \hbox{ Hadamard gate}
          \end{align*}
          Why unitary? $\ket{\phi} \to \ket{\psi} = U\ket{\phi}$ so $\norm{U\ket{\phi}} = \norm{\ket{\phi}}$ for all $\ket{\phi}$.

    \item[Postulate 3.1] The time evolution of a closed system is given by Schr\:{o}dinger's Equation
        \[i\planck \frac{d\ket{\psi(t)}}{dt} = H\ket{\psi(t)}\]
        $\bra{\psi}H\ket{\psi}$ is the expected energy.
        \begin{enumerate}[label=\arabic*)]
            \item $H(t)$ time dependent
            \item $H(t) \identically H$ time independent
        \end{enumerate}
        $\ket{\psi(t)} = e^{iHt}\ket{\psi(0)}$

        $H$ hermitian $\implies$ $e^{iHt}$ is unitary

        For example: $H = \sum_{E} E\ket{\phi_E}\bra{\phi_E}$\quad $\ket{\phi_e}$ energy basis

        $e^{iHt}\ket{E} = e^{iEt}\ket{E}$\quad$\ket{\phi} = \sum a_E \ket{E}$\quad $e^{iHt} = \sum a_E e^{iEt}\ket{\phi_e}$
    \item (Composite System) The state of a composite system is a tensor product
          of the state space
\end{enumerate}

\begin{definition}[Tensor Product Space]
    Given $V = \complexnums^{\vecspace{X}}$ and $W = \complexnums^{\vecspace{Y}}$,
    \begin{align*}
        V \tensorprod W = \complexnums^{\vecspace{X} \times \vecspace{Y}} = \set{\phi \suchthat \vecspace{X} \times \vecspace{Y} \to \complexnums}                    \\
        \phi \tensorprod \psi(x, y) = \phi(x)\psi(y)                                                                                                                  \\
        \ket{\phi}\tensorprod\ket{\psi}                                                                                                                               \\
        V \tensorprod W = \spanset\set{\ket{\phi} \tensorprod \ket{\psi} \suchthat \ket{\phi} \in V, \ket{\psi} \in W}                                                \\
        \alpha\left(\ket{\phi}\tensorprod\ket{\psi}\right) = \alpha\ket{\phi}\tensorprod\ket{\psi} = \ket{\psi}\tensorprod\alpha\ket{\psi}                            \\
        \alpha\ket{\phi_1}\tensorprod \ket{\phi} + \beta\ket{\phi_2}\tensorprod\ket{\psi} = \left(\alpha\ket{\phi_1} + \beta\ket{\phi_2}\right)\tensorprod \ket{\psi} \\
        \left(\bar{\phi_1}\tensorprod\bar{\phi_2}\right)\left(\ket{\psi_1}\tensorprod\ket{\psi_2}\right) = \bracket{\phi_1}{\psi_1}\bracket{\phi_2}{\psi_2}
    \end{align*}
    Given $\set{\ket{v_i}} \subseteq \vecspace{V}$, $\set{\ket{w_j}}\subseteq\vecspace{W}$ orthonormal bases,
    the set $\set{\ket{v_i}\tensorprod\ket{w_j} \suchthat 1 \leq i \leq n, 1 \leq j \leq m}$ is an orthonormal basis
    of $\vecspace{V} \tensorprod \vecspace{W}$. So $\dim\left(\vecspace{V} \tensorprod \vecspace{W}\right) = \dim\vecspace{V}\dim\vecspace{W}$
\end{definition}

\begin{example}[Product State]
    Take $V_1 = V_2 = \complexnums^2$. Then
    \begin{align*}
        \ket{0} \tensorprod \ket{0} & = \ket{00} \\
        \ket{0} \tensorprod \ket{1} & = \ket{01} \\
        \ket{1} \tensorprod \ket{0} & = \ket{10} \\
        \ket{1} \tensorprod \ket{1} & = \ket{11} \\
    \end{align*}
    And
    \begin{align*}
        \ket{\phi}                        & = a\ket{0} + b\ket{1}                                                 \\
        \ket{\psi}                        & = \alpha\ket{0} + \beta\ket{1}                                        \\
        \ket{\phi} \tensorprod \ket{\psi} & = a\alpha\ket{00} + a\beta\ket{01} + b\alpha\ket{10} + b\beta\ket{11}
    \end{align*}
\end{example}

\begin{example}[Entangled State]
    \begin{align*}
        \ket{\Phi^+} & = \frac{1}{\sqrt{2}}\left(\ket{0}\tensorprod\ket{0} + \ket{1}\tensorprod\ket{1}\right) \\
                     & = \frac{1}{\sqrt{2}}\left(\ket{00} + \ket{11}\right)                                   \\
        \ket{\Phi^-} & = \frac{1}{\sqrt{2}}\left(\ket{00} - \ket{11}\right)                                   \\
        \ket{\Psi^+} & = \frac{1}{\sqrt{2}}\left(\ket{01} + \ket{10}\right)                                   \\
        \ket{\Psi^-} & = \frac{1}{\sqrt{2}}\left(\ket{01} - \ket{10}\right)
    \end{align*}
\end{example}

The vectors $\ket{\Phi^+}$, $\ket{\Phi^-}$, $\ket{\Psi^+}$, $\ket{\Psi^-}$ form an
orthonormal basis of $\complexnums^2 \times \complexnums^2 \isomorphic \complexnums^4$. The state
is called \emph{entangled} because $\ket{\Psi^+} \neq \ket{\phi} \tensorprod \ket{\psi}$ for any
$\ket{\phi}$, $\ket{\psi}$, i.e., it cannot be expressed as a product state.

Operations on Tensor Product System
\[A \in \boundop{\complexnums^n} \quad B\in\boundop{\complexnums^m}\]

Define $(A\tensorprod B)\ket{\phi}\tensorprod\ket{\psi} = A\ket{\phi}\tensorprod B\ket{\psi}$.

\[A \tensorprod B \in \boundop{\complexnums^n \tensorprod \complexnums^m} = \spanset\set{A \tensorprod B \suchthat A \in \boundop{\complexnums^n}, B\in\boundop{\complexnums^m}}\]

\begin{definition}[Product Measurement]
    \[A \in \boundop{\complexnums^n} \quad B\in\boundop{\complexnums^m}\]
    \begin{align*}
        (A \tensorprod B)^* = A^* \tensorprod B^* = A \tensorprod B\hbox{ is Hermitian in $\boundop{\complexnums^n \tensorprod \complexnums^m}$} \\
        \bra{\phi}\tensorprod \bra{\psi}\left(A \tensorprod B\right)\ket{\phi}\ket{\psi} = \bra{\phi}A\ket{\phi} \bra{\psi}B\ket{\psi}
    \end{align*}
    $\bra{\phi}A\ket{\phi}$ is the expectation of observable $A$ given $\phi$ and
    $\bra{\psi}B\ket{\psi}$ is the expectation of observable $B$ given $\psi$.
\end{definition}

Given POVMs $E_i$ and $F_j$, we have
\begin{align*}
    \sum E_i                 & = I_1                 \\
    \sum F_j                 & = I_2                 \\
    \sum E_i \tensorprod F_j & = I_1 \tensorprod I_2
\end{align*}
and $E_i \geq 0$, $F_j \geq 0$, and $E_i \tensorprod F_j \geq 0$.

\begin{definition}[Partial Measurement]
    Given $A \in \boundop{\complexnums^n}$,
    \begin{align*}\bra{\phi}\tensorprod\bra{\psi} A \tensorprod I \ket{\phi}\tensorprod\ket{\psi}
         & = \bra{\phi}A\ket{\phi}\bracket{\psi}{\psi} \\
         & = \bra{\phi}A\ket{\phi}
    \end{align*}
    or
    \[\sum_m E_m = I \quad \bra{\phi}\bra{\psi}(E_m \tensorprod I)\ket{\psi}\ket{\phi} = \bra{\phi}E_m\ket{\phi}\]
\end{definition}

\begin{align*}
    \hbox{Product State}   & \sim \hbox{independent measurement outcome} \\
    \hbox{Entangled State} & \sim \hbox{correlated measurement outcome}
\end{align*}

\begin{example}
    Given
    \begin{align*}
        E_0          & = \ket{0}\bra{0}                       \\
        E_1          & = \ket{1}\bra{1}                       \\
        E_+          & = \ket{+}\bra{+}                       \\
        E_-          & = \ket{-}\bra{-}                       \\
        \ket{\Phi^+} & = \frac{\ket{00} + \ket{11}}{\sqrt{2}}
    \end{align*}
    Then
    \begin{align*}
        \bra{\Phi^+}E_0\tensorprod I\ket{\Phi^+}                 & = \frac{1}{2} = \bra{\Phi^+}I \tensorprod E_0\ket{\Phi^+}    \\
        \bra{\Phi^+}E_1\tensorprod I\ket{\Phi^+}                 & = \frac{1}{2} = \bra{\Phi^+}I \tensorprod E_1\ket{\Phi^+}    \\
        \bra{\Phi^+}E_i \tensorprod E_j\tensorprod I\ket{\Phi^+} & = \begin{cases}\frac{1}{2} &\hbox{ if $i = j$, not independent on $i$, $j$}\\0 &\hbox{ if $i \neq j$}\end{cases}                                 \\
        \bra{\Phi^+}E_i\tensorprod E_{\pm}\ket{\Phi^+}           & = 0 \hbox{ for $i = 1, 2, \dots$}                            \\
        \bra{\Phi^+}A \tensorprod I\ket{\Phi^+}                  & = \frac{1}{2} \bra{0}A\ket{0} + \frac{1}{2}\bra{1}A\ket{1}   \\
                                                                 & \neq\bra{\phi}A\ket{\phi}\forall \ket{\phi}\in\complexnums^n
    \end{align*}
\end{example}

The observation of $\ket{\Phi^+} \in \complexnums^n \tensorprod \complexnums^n$ on the first system does not match
any vector state. Does this violate postulate 1?

No, because a \emph{closed system} has a vector state $\ket{\phi}$. The state of a general
system is given by a \emph{density operator}.

A vector state $\ket{\phi} \in \complexnums^n$ is also called a \emph{pure} state.

In general, a quantum system can have mixed state, given by $\set{p_i, \ket{\psi_i}}$, an
\emph{ensemble} or pure states. $p_i$ is the probability the system is in $\ket{\psi_i}$.
Then \[P = \sum p_i\ket{\psi_i}\bra{\psi_i}\]
is the \emph{density operator}.

\begin{example}
    If $P = \ket{\phi}\bra{\phi}$ \dots %TODO can't read the notes
\end{example}

\begin{enumerate}[label=Postulate \arabic*]
    \item (State) The state of a quantum system is completely determined by a density
          operator $P$ acting on the state space of the system
          \[P = \sum p_i\ket{\psi_i}\bra{\psi_i}\]
          if the system is of probability $p_i$ in the pure state $\ket{\psi_i}$.
    \item (Measurement) A POVM $\set{E_m}$ has probability of
          \[\prob{m} = \trace(PE_m)\]
          to be outcome $E_m$.

          (Observable) The expected value of an observable $A = A^*$ in a given
          state $P$ is $\trace(PA)$.
    \item (Evolution) The unitary evolution of a closed quantum system is
          given by \[P \to UPU^*\]
\end{enumerate}